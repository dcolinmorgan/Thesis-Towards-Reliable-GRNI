% \externaldocument{4/chapter4.tex}
\chapter{Present Investigations}

\section{GeneSPIDER - GRN inference benchmarking with controlled network and data properties (PAPER I)}
GeneSPIDER is a package developed for MATLAB to offer an outlook-proof environment for comparing inference algorithms. Some 15 modern inference methods are included and any number more can be easily added to this end. This benchmarking ability is made possible by a synthetic network and dataset creation pipeline with the ability to tune many properties therewithin. This informs the user when analyzing experimental dataset which methods and settings are appropriate for optimal inference. 
The controlled creation of dataset and networks of various sundry properties allows for an unbiased appraisal of any given method's accuracy in data-based network reconstruction (via \crefrange{eq:MCC}{eq:AUPR}, among others). Data properties owing to this performance can then be picked out when parameters are satisfactorily varied, and used to inform inference when gold standard networks are not available for accuracy measure, \ie inferring networks from biological dataset. The environment can also inform the scientist of experimental properties such as replicate number which will make downstream network inference more reliable and accurate.
A benchmark of methods across SNR, topology, size, and condition number showed many methods struggle to infer a network with accuracies better than random (50\% accuracy) when the SNR is significantly low, \ie low 0.01. This should lead scientists to strive for ever more higher fidelity transfers of genetic information to quantized datasets ultimately feeding into computer models, \ie through more experimental replicates. It also lead us to investigate ways of boosting inference accuracies by modeling the randomness many times to negate its effects, as well as weighing the costs of removing some genes shown to be especially noisey.

% \subsection{Outlook}

A major effort has already been made to incorporate more methods into the toolset. However, until these tools are freely available to the larger community of bioinformaticians, they will remain largely underutilized. A minor effort has been made to opensource this package to python, and a greater push is now needed if it ever hopes to gain adoption.


\section{A Generalized Framework for Controlling FDR in GRN Inference (PAPER II)}

NestBoot applies a bootstrapping protocol to any inference method to assess the stability estimated support values in order to mount a challenge to the many challenges uncovered in the GeneSPIDER benchmark, namely improving inference accuracies under poor SNR such as biological datasets. NestBoot inference is based on comparison of inferred networks from measured data to those inferred based on randomized data, which provides a sense of attaining such a network by chance. This allows for the control of FDR in a highly conservative formulation by comparing bootstrap support values to those of the same pipeline fed with shuffled data. This approach saw increases in regression methods: LASSO, LSCO and RNICO as well as tree consensus method Genie3 and the MI based CLR. across every SNR level, although increases were not uniform in their step.

% \subsection{Outlook}
The initial NestBoot protocol was implemented in the few methods benchmarked in the GeneSPIDER paper. However, they were restricted in many ways and quite heavy computationally. Furthermore, as the number of methods expanded, so too has the need for a more universal nested bootstrap adapter. A newer version now contains parameters to enter any method incorporated to GeneSPIDER, as well as a few methods which attain major speed boosts when operated over GPU utilizing native MATLAB CUDA functions. 

\section{Perturbation-based gene regulatory network inference to unravel oncogenic mechanisms (PAPER III)}

Our previous investigations of inference performance suggest optimal experimental design of many replicates mixed with partial knockdown. To this end, 40 genes known to have some involvement in cancer progression were perturbed with siRNA in triplicate. This allowed for network inference using three current methods. Accuracies were determined through a comparison not to true gold standard network, since none exist, but instead in a leave out manner to the original and validation datasets of the same gene compositions. This was accomplished by minimizing the error both of experiment read and inference, \ie E and F errors on Y and P, respectively. Since error can only be estimated from variances among Y matrix replicates, balancing of error was done to estimate inference errors, F.

Here, a common set of genes was perturbed and measured independently in human squamous carcinoma cell line. The training dataset contains genes perturbed and measured three times as experimental replicates, while the validation dataset contains the same genes perturbed in pairs without replicate. Taking into account various data properties, the training dataset was used to infer a network of the underlying mechanisms of control. This network was able to reproduce its training data in a leave out manner, and whatsmore, it is robust enough to reproduce a separate validation dataset to a degree of accuracy higher than expected by chance. In this way, many known links were recovered during the inference, as well as novel links proposed, two of which were verified experimentally. 

% \subsection{Outlook}
A major contribution of this paper is in the form of the knockdown dataset, performed some years ago on the technology of that time. Today, a incredibly powerful technology has been discovered and developed for targeted knockdowns far more precise than siRNA. CRISPRi offers the targeted knockdown of siRNA without the various off-target effects, and at a reduced cost. For the cost of personnel time to design primer sequence tags and carry out interference protocol as well as ordering primers the experiment can be done in multiplex, creating a new dataset of many more replicates for the same cost as the triplicate siRNA experiments of days past.


% \section{Large-scale Gene Regulatory Network Inference Reveals General and Specific Oncogenic Mechanisms (PAPER IV)}

% The L1000 offers a trove of richly characterized gene perturbation data, singly knocked down on a scale much larger than previously investigated here. Inferring a gene regulatory network (GRN) from this data grants insight into specific mechanisms directing cellular behavior in each of these disparate cell types. Networks are inferred from these 9 human cancer and 2 stem cell lines to explore the differences among human cancer GRNs. This is carried out in a reliable way using the NestBoot inference protocol for inferring individual cell line GRN by restricting the inclusion of false links. A comparison between these 2 healthy samples and cell line specific networks formed among these 9 cancer subtypes is made to find regulatory elements common among cancer types. Finally, an jacardian all-versus-all comparison is used to contrast specific regulators within each cell line to those in the healthy network. The results confirm many relational elements per specific oncogenic subtype, as well as novel regulations predicted within each cell line. However, up to this point we see a disparity in the degree distribution per inference method, namely that where LSCO presents mega-hubs, nodes with many outgoing regulatory links. This appears to reliably occur with N>100, where such gene lists would find unreliable relations inferred under such inference. This is an ongoing study and presents many mathematical questions. %Such findings made in the reliable fashion offered under NestBoot offer keys to identifying universal pathways involved in cancer progression and possibly even progenesis.

% % \subsection{Outlook} 

% Verifying such novel, inferred interactions would obviously be the first step in validating such reverse engineering. Further, utilizing the drug perturbation information for a screen of potential drug repositioning is a likely source of novel insight into the known and unknown treatments of compounds already in circulation amongst the world population. A primary outcome of interest would be in identifying drugs which inversely target the same patterns perturbed under certain cancers. Such would signals a dualism in the cause of the disease and its potential treatment, \ie the antagonistic relations targeted by each. Furthermore, finding drugs which actually have the same effect as disease would be an interesting discovery in so far as creating model organisms for outlook study. A simple dosage could present similar enough symptoms to offer manageable phenotypic models for developing treatment. 

% Tools comparing networks beyond n{\"a}ive link-link analysis are powerful and a perfect fit here. Specifically, ALPACA seems a great compliment and expansion of the present work, where networks inferred from different cell lines describing different tissue derived cancers are compared for module similarity. Further expansion could see MONSTER used to find transitions between time points recorded in the L1000 dataset, thought presently not containing inferred networks. Possibly state transitions between distinct cell line networks describing the same cancer could be estimated as well.


\section{A Subset Selection Method for Accurate Gene Regulatory Network Inference from Uninformative Datasets(PAPER IV)}

The L1000 offers a trove of richly characterized gene perturbation data, singly knocked down on a scale much larger than previously investigated here. Inferring a gene regulatory network (GRN) from this data grants insight into specific mechanisms directing cellular behavior in each of these disparate cell types. We used the L1000 data where roughly 978 genes were perturbed and subsequently expression levels quantified in 9 cancer cell lines. Key properties of the datasets, namely, signal-to-noise ratio (SNR) and condition number which we have shown to affect the performance of various inference methods were identified with the idea to maximize performance by optimizing each factor. 
In order to improve the poor SNR of the dataset we developed a gene reduction pipeline which eliminates the uninformative genes from the system using a selection criteria based on SNR until reaching an informative subset. We present a pipeline which identifies an informative subset in an uninformative dataset, improving the accuracy of the network inference.
This is quite opposing the quest to infer genome-scale GRN and possibly the best practices of data science regarding discarding data. However, until methods are created which can adequately handle noise on this scale (not an insurmountable challenge for \textbf{Paper II}), this is a first step in the direction of isolating probable submodules within a greater GRN yet to be uncovered.