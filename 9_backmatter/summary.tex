% Thesis summary -------------------------------------

% Should be in swedish if thesis are in another language, otherwize in english.
\selectlanguage{swedish}

%\summarytitle{Summering}

%\begin{summary}        %this creates the heading for the declaration page
\chapter{Sammanfattning}
Ur en ursprunglig, oenhetlig oordning, verkar ordning upprepade gånger ha uppstått i universum som en effekt av fysikens lagar. Där entropin hade föredragit att tomrummet förblev tomt, började kroppar smälta samman i allt större utsträckning, vilket ökade ordningen. När allt tyngre ämnen började fylla tomrummet, skapades en tillfällig stabilitet som motverkade entropins slutgiltiga mål genom att sprida ut materia här och där. Över tid ersattes ordning av större ordning, ända till den största triumfen över entropin skedde: livets uppkomst! (Eller det verkade så till en början: det har nyligen argumenterats för att liv istället ökar den samlade entropin fortare än frånvaron av liv skulle göra \citep{england2013statistical}) Sedan uppkomsten av replikation och självreplikerande biomolekyler har antalet hoptvinnade relationer ökat häpnadsväckande, och varje ny anpassning ger nya användningsområden för nya och gamla beståndsdelar, levande såväl som icke-levande (dvs konkurrens, symbios, parasitism etc). Dessa relationer existerar inom nästan varje nivå av liv, från djurplanktons avhängighet på tidvatten och månen till vårt samhälles beroende av uråldriga reserver av kolbaserad energi; återigen, ingenting i naturen existerar i ett vacuum.


Sådana typer av relationer kan kartläggas, katalogiseras och analyseras med hjälp av nätverk. Ett nätverk fångar upp flexibiliteten hos kraftfulla system in i en enda struktur; detta betyder dock inte att nätverk innebär förenkling; faktum är att många nätverk är så kompakta av förbindelser att de själva motsätter sig tolkning, just såsom de system som de beskriver \citep{dianati2016unwinding}. Varje system som innehåller två komponenter kan summeras som ett nätverk, med varje komponent sedd som en nod och deras relation till varandra som en länk. Olika vikter och konstnärliga utsmyckningar kan ges både till noder och länkar för att öka den kombinerade densiteten av information, men det är själva modellen av det totala systemet som ger mening till nätverket. Till exempel, när alla länkar är inräknade kan man ana den sociala påverkan av en nyhet, hur rikedomar sprider sig mellan släkter i utvecklingsländer, och av speciellt intresse här, hur gener fullgör instruktionerna som finns kodade i vår livskod. Vad händer när en specifik gen nedregleras – finns det då något annat sätt som gör att den kommer till uttryck, eller är systemet för alltid förändrat, dömt att anpassa sig eller till att helt enkelt ha mist sin funktion? Och när kombinationen av sådana reaktionsvägar förändras, hur har systemets överlevnadsinstinkter förberett det? Biologiska system av alla storlekar har visat sig vara mycket stabila, vilket man intuitivt kan gissa från vidden på mänsklig föda eller bredden av människor, djur och olika språk på denna planet. Genregulatoriska nätverk (GRN) är inget undantag. Faktum är att mäta en gens relation till en annan, dvs kartlägga dess lokala nätverk, är extremt svårt just på grund av detta, att relationerna passerar många mellanspelare och ofta är sammansatta genom unika loopstrukturer som förstärker deras relation. 


Verktyget GeneSpider i \textbf{Paper I} försöker erbjuda en viss tydlighet för att lösa denna uppgift, genom att ge lika villkor för att testa många olika metoder av nätverksinterferens, i hopp om maximal pålitlighet. Verktyget tillåter  generation av  syntetisk data vilken speglar många egenskaper som finns i verklig experimentellt framtagen biologisk data. Denna data tillåter full kontroll i skapandet av nätverk , något som ofta saknas i faktiska experiment och ger en uppskattning av noggrannheten. Nutida metoder av nätverksinterferens varierar på många olika sätt, ingen mindre trivial än dess anslag av variation i mätningar. Vårt ramverk för FDR-kontroll i nätverksinferens i \textbf{Paper II} mäter denna variation inom frekvenser tillräckliga noga för att ge den underliggande inferensmodellen  en god uppskattning av den inneboende variabiliteten för systemet. Vilket ger tillbaka ett pålitligt nätverk baserat på strikta kriterier för att acceptera falska nätverkslänkar. Den störnings-baserade interferens metoden presenterad i \textbf{Paper III}  studien är kulmen av denna och annan forskning, alla som guidats av experimentell design, genom inkluderandet av många replikat av individuellt brus. Dessa data behandlades i verktyget ”FDR restricing framework” för att ge ett pålitligt nätverk som mäts mot en strikt korsvalideringsmetod. På samma sätt kan användandet av dataegenskaper för att begränsa inferens till endast den som kan uttydas med rimlig noggrannhet möjligen leda till en större acceptans inom fältet (\textbf{Paper IV}) och därmed mindre motstånd när man går vidare till den kliniska domänen. Att ta bort mycket experimentell data kommer onekligen skapa frustration men det kan också leda till en revolution för större experimentellt djup inom biologisk karakterisering. \\


\noindent{översatt av Malin Lundahl och Sofie Wendel, med ytterligare redigering av Thomas Hillerton.}



%\end{summary}

\selectlanguage{english}

% ----------------------------------------------------------------------