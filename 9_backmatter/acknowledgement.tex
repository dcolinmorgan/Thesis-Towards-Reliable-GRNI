% Thesis Acknowledgements ------------------------------------------------


%\begin{acknowledgementslong} %uncommenting this line, gives a different acknowledgements heading
%\begin{acknowledgements}      %this creates the heading for the acknowlegments
\chapter{Acknowledgements}

I would not have been able to complete this work without my immediate family. My brother Evan visited the first and third summer of my study, the second of which trip lasted nearly 10 weeks as he studied during the day and joined in climbing and exploring the city by night. His strength and unwavering love and encouragement made the summers that much more exciting, as have video chat these past years made the dark winter months that much more bearable. My mother Cynthia and father Thomas provided much the same. From their encouragement I strove to continue studying year after year rather than jump to industry and the constraints of \emph{real adult life}. Even as they pushed to be a financially responsible adult and consider more conventional jobs in industry, I clung to academia in a move towards quite the opposite. More than that, my mother taught me early on teaching that good arguement is key to logical discourse. That it's not enough to make sense to yourself but that communication is check against insanity, not to mention spreading good ideas. For his part, my father taught me to look up into the canopy of a tree as you walked by, to marvel at the everyday and question its being. Second only to this family is my neighborhood family, the Osbornes: Big and little Larry, Kathy and Meredith (and now Adam and Clark), have been friends, siblings, parents and second or extended family.
%%Core Sonnhammer Group
Toward the matter of completing the work, I must thank Erik Sonnhammer above all, for giving me the opportunity and time to learn and grow as a scientist, ascending quite the learning curve toward our shared goal of contributing to the field. Equally, I must thank Sweden for investing in a foreign born son who has been made to feel very much welcome and part to the society and culture. I took the position never having breached the confines of the continental U.S., and only after four years do I realize I could not have found a more welcoming new home. It is amazing such financing and support for foreign scientists exists in a world so wrought with divisiveness that the mirror process in my home nation is far from common. It is my intention and indeed my deepest hope to one day return and repay this wonderfully open society for its investment. Also to my co-supervisor Torbj{\"o}rn Nordling for hours of one-on-one tutoring, both in person and via skype. Visiting his young lab filled with bright-eyed undergraduate and masters students was one of the highlights of both my research and personal life, solving a crucial piece of the validation in \textbf{Paper III} as well as meeting many new hiking, travel, street workout, late-night eating and generally adventurous friends in the process. Similarly, this research is heavily indebted to that of Andreas Tj{\"a}rnberg, whos guidance during the early days of my GRN life proved invaluable to my understanding and ultimately to these contributions. A latecomer to the group, Deniz Seçilmiş seemed bound to be to me what I was to Andreas, yet, she has surpassed this initial estimation in almost every way, teaching me so much along her way toward becoming a self sufficient methodologist. I anticipate future collaboration with this core team as I very much hope to maintain regular communication as we have these past years, delving deeper into the possibilities contained within the GRN field and beyond.
%%Other Sonnhammer Group
Friendships from within the Sonnhammer group at large has been a major source of encouragement along this road, including friendships I hope to last my lifetime. Namely that of Christoph Ogris and his beautiful wife Lisi, who took me in much like a lost puppy during my first harsh Swedish winter, sharing with me their passion of rock climbing, hiking and general enthusiasm for all things natural. Their relationship is another example to my eyes of determination in ones life, being decisive in intent then following up to make sure it works out. I would not be who I am today without their friendship. Furthermore, seeing Christoph in his natural habitat (the Austrian Alps of Schladming) gives me a better appreciation for the simple pleasure to be had over dinner, wine, fire and snow with friends in nature. In those early puppy days I looked up to Christoph like an older brother, and now that we have both summited mount doctorate I hope that our shared experience proves a bond all the stronger. They even introduced to now mutual friends Roman and Sandra, Swiss Chris and Maja. Similarly like being amongst family, the experience being welcome into the homes of both Deniz and Miguel during is something I will treasure. I gained such an appreciation for these amazing individuals seeing them in their respective \emph{natural elements} after having known them in the abstraction that is the (computational) lab environment. Similarly yet of a unrelated variety, Mateusz Kaduk and his bride to be Kate have offered wonderful friendship upon lab events and group outings in the city, most memorably visiting the Skansen for each beautiful pagan spring ritual derived from witch burning of olde. As a house-mate and long time Stockholm resident, Stephanie provided much startup help for my moving process as well as a wonderfully unique perspective on all things German and Soviet, both past and present, not to mention the swimming classes she gave for Lisi, I and universtiy students. Lest I forget a major inspiration not just for science but for living life, Dimitri Guala and his relationship with his beautiful wife Izabela have shown what professional careers outside academia can look like and the lifestyle they afford, not just monetarily but in values of health and family. 
%%SciLifeLab
To my first lab friends, Annemarie Perez Boerema and Miroslav Huliciak, both of whom opened their homes to me for visits during various holidays. To Daniel Jans, David Jess for the many lunch time runs, and to the many other lab friends who would come to occupy the halls of our beloved Gamma 3, I thank you for your endless conversations, pondering each member's home country's latest election results, sharing fikas, cakes and beers at the pub night: Marta Carroni, Juni Andrell, Alex Muhleip, Narges Mortezaei, Victor Tobiasson, Jose Miguel de la Rosa Trevin, Giovanna Coceano, Francesca Pennacchietti, Jonatan Alvedlid, Andreas Boden, Steven Edwards, Liang Zhang, Evgeny Akkuratov, and a few of the floor's other P.I.'s Ilaria Testa and Alexey Amunts, the later who's application expertise helped me attain my first postdoc position. Other SciLifeLab \& DBB friends Mirco Michel, John Lamb, Marco Salvatore. 

Lest I forget my first contact in Sweden outside the lab, the woman who picked me up at the airport after my first international flight, whos brother drove me back into the city after I purchased a bike in disrepair via the pendeltag, who overlooked the security deposit when I was broke after arriving in Sweden a month before my salaried contract started; Ulrika Larsson was like my Swedish mother those first two years, and I cant imagine the hardship she saved me from during that time. Futhermore, to the Swedish people, firstly for enabling such a healthy period of learning, especially to a foreigner, through more than adequate funding; not to mention for making learning easy by practicing such good English that it is my great shame to have learned more Turkish over a ten day trip in 2018 that I have Swedish in 4.5 years.
%%Klatter
I thank the many climbing friends I have made inside the various klattercentrets as well as outside in the innumerable afterwork and weekend session throughout Stockholm and southern Sweden, many of whom I also shared scientific discussion as fellow students. Namely, Jacupo Fontana (another larger than life big brother type in my eyes) and his GF Franchesca, Giacamo Sitzia, Kaveh Rezania, Leo Sparring, and the dozen or so others I may have forgotten.
%%OSU
To my Ohio State advisor Kun Huang, lab PI James Chen, professors Philip Payne and Albert Lai, and program counselor James Gentry who taught me the importance of paperwork efficiency. To my best classmate from this time, Marcelo Lopetegui Lazo and his wife Barbara, for their friendship, Chilean sentiment and fun times, skiing and partying. To my Ohio State friends Alan, Amy, Caleb, Carrie, Kyle, and Steve, for the endless nights discussing politics over mead, dancing out at concerts and generally cavorting through central Ohio. Many have visited me in Stockholm and abroad, always bringing an element of home along with them. To my best roommate, Jatin Gupta, who was an ideal role model of simplicity and loving life during his studies, not to mention the best chef in the area! To my OSU hockey brothers, especially Dane, Manuel, Eric and my closest bud Big John. He has introduced me to many of the luxuries of independent adulthood, least meeting new friends David and Christian while biking our way around Berlin. And to my college roommate Jeremy Verner, for reminding me that work ethic can being liberating rather than constraining. To my college hockey and rowing friends, as well as my high school rowing and rugby friends, for never letting me forget that pain does not have to be ones enemy.
%%BUD, DEB
To the memory of my grandfather Bud, whom this writing is dedicated to, who been a major force in my life without the two of us ever having met. A force of inspiration is key to my development, and the only equal in my life would be that of his second daughter's best friend, my aunt Debra, who has been my scientific role model since my first science fair entry (on geotropism, taking home second prize with the help of my father). My mother has long told me tales of her doctoral studies, which I carry with me and share retell to friends under stress, that tackling just ``one mouse a day'' day in and day out can yield grand accomplishments. And to her husband Fred who Fred who scaring me into realizing foreign institutions compete for same journal publications. To my grandmothers Helen and Irene, whos refrigerators and candy bowls were never empty despite having 15+ respective other grandchildren regularly ransack their homes, and to my grandfather Bob, who showed me such joy and friendship early on as his "buddyboy".

For all these and more wonderful relationships I am forever grateful; each means more than I could have imagined when I set out on this journey four and 26 something years ago, and I only hope for their continued prosperity.

And last but not least, an extra special thanks to Miguel... for being Basque.
%\end{acknowledgements}
%\end{acknowledgmentslong}

% ------------------------------------------------------------------------


