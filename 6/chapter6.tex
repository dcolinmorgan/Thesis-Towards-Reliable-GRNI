% this file is called up by thesis.tex
% content in this file will be fed into the main document

%: ----------------------- introduction file header -----------------------
\externaldocument{1/introduction.tex}
\graphicspath{{6/figures/}} % specifies where the figures are stored

% ----------------------------------------------------------------------
%: ----------------------- content ----------------------- 
% ----------------------------------------------------------------------

% \chapter{\label{ch6}Afterward} % top level followed by section, subsection

\chapter{\label{}Afterwards}

% \subsection{\label{}Name of subsection}

Boltzmann showed that entropy is what we really count when we delineate units of time. Furthermore, it has recently been proposed \citep{connes1994neumann} that this increase of entropy we envision for the universe is simply a bias toward the order which we have evolved to identify and thus find meaning in. Rovelli proposes that the future may be no less ordered, that there is no change in total entropy. Order simply manifests in new ways, as in binning socks by color, only to have a colorblind man later bin them by length.

The continuum we inhabit calls for modeling nevertheless, not for the sake of time but to learn the behavior and bounds of the continuous change of life. For such steady-state models as presented here, cell cycle synchronization is key, to measure seperate experiments through similar periods in their development, then presumably under similar GRN regulation. This can be accomplished through starvation, ie media deprivation, to force cells to recover from same stimuli together and thus bring their growth phases into sync. From such a beginning, time-series experiments would then enable quantification beyond simple assumptions of steady-state, enabling modeling of relationship permanency, \ie if links are static, or more likely, if they are as transient as the condition life finds itself inhabiting. This is one of the capabilities we have provided in the vastness of the L1000 dataset. Its high-throughput multiplexing has many time points taken within cell lines, which can feed into a model of network evolution, how links come into being, and thus a potential roadmap for which network conditions preclude any given network outcome.
%%drugL1000
The LINCS consortium also provides in its L1000 portal small molecule and drug induced perturbation experimental measures in keeping with the shRNA measures we use in \textbf{Paper IV}. It would be very interesting to develop a method for modeling drug perturbation on a network template, which could then be cross-correlated with inverse regulatory interactions in disease. Such pairing of disease and drug information in the context of networks could provide not only novel drug gene-targets, but also open the ability of less specific drugs to target general network modules to achieve the same aim of changing the regulation of key disease pathways.

One straightforward method to achieve this using the same models used here would be to code perturbation design onto small molecule, drug induced perturbation by way of the STITCH database \citep{kuhn2009stitch} or DrugBank \citep{wishart2006drugbank}. Creating such network models based purely on replicate, multiple perturbation experiments could uncover novel uses for drugs which have passed FDA phase I and II safety testing, thereby bypassing many years of safety testing not to mention chemical/structural development \citep{oprea2011drug}. These could then be very easily validated, at least in a rudimentary way through IC50 assay on the various standardized cell lines.
%%NN inf
Beyond perturbation-based inference methods, classic perception neural networks may hold key to further enhancing inference accuracy. Assembling a few layers to account for ingoing and outgoing links, strength as weights and up or down regulation would be easy when matched with a bank of sufficient training data. GeneSPIDER (\textbf{Paper I}) could be such a key, allowing for the creation of much synthetic data for training, with hold out allowing for accuracy evaluation before feeding in real biological data. Furthermore, a bootstrap method drawing from experimental replicates would increase the training data amount as well as feed in a level of robustness where a network can describe many datasets due to varying levels of noise, combinations of experiments.

% \noindent\rule{12cm}{0.4pt}

These past years I have been inspired to study many of the ideas present here without directly identifying them as such. I enjoyed several popular non-fiction and fiction works over the past year I spent compiling these chapters. These works reinforced what might be obvious, the idea that genes do not and cannot exist in isolation and any system-altering force goes against the investigative process. I saw the repercussions in several different scientific domains concomitantly coming to understand this principle. The first I have referenced formally as Carlo Rovelli's Order of Time\cite{rovelli}, which I followed reading Nick Pyenson's Spying on Whales\cite{pyenson},  and then Richard Powers' The Overstory\cite{powers}. Most notably, this prospectus suggests a position to the much belabored quandry ''nature vs nurture'', which would seemingly posit that, like the idea that time is an artificial construct of the human mind or the age-old question ''what is the meaning of life'', there can be no seperation of the two, nature and nurture, and these are simply bad questions. As large challenge yet overcome is modeling disease \emph{in vitro}, as any such perturbation of environment, \ie culturing practices no matter how stringent the protocol, alters the system under study. Following, studying a tree isolated from its forest brethren does not characterize the wild-type individual nor does investigating an orphaned and abandoned whale calve in captivity capture its true nature. Such investigation, not excluding that presented here, often fall short of generalizing to the question initially posited, and indeed call upon tests of statistics to worm their way back into meaningful investigations from the abstractions they become. The system is composed of individuals which they themselves are reflections of the constrains provided by the environment. More to the point, there is no suitable nature without a nurture for it to grow and thrive within, the two are inextricably linked, as previously alluded to in the very beginning of \cref{sec:intro}.

% I have been inspired by the robust relationships present in our natural world, in the way forest root systems propagate between individuals to form reactionary defenses against invasive species and how whale pods can communicate long term relationships among individual members...\\

% Future work hopes to involve the akaike and/or bayesian information criterion (AIC/BIC) to further bolster our results away from any bias we may introduce in our current estimations of inference performance, etc.


% ---------------------------------------------------------------------------
%: ----------------------- end of thesis sub-document ------------------------
% ---------------------------------------------------------------------------

