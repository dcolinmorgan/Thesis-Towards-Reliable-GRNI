% this file is called up by thesis.tex
% content in this file will be fed into the main document

%: ----------------------- introduction file header -----------------------
% \externaldocument{1/introduction.tex}
\graphicspath{{6/figures/}} % specifies where the figures are stored

% ----------------------------------------------------------------------
%: ----------------------- content ----------------------- 
% ----------------------------------------------------------------------

% \chapter{\label{ch6}Afterward} % top level followed by section, subsection

\chapter{\label{}Afterwards}

% \subsection{\label{}Name of subsection}

Boltzmann showed that entropy is what we really count when we delineate units of time. Furthermore, it has recently been proposed \citep{connes1994neumann} that this increase of entropy we envision for the universe is simply a bias toward the order which we have evolved to identify and thus find meaning in. Rovelli further postulates that the future may be no less ordered, and that there is no change in total entropy \citep{rovelli}. Order simply manifests in new ways, as in binning socks by color, only to have a colorblind man later bin them by length.

A construct of the human mind or not, the continuum we inhabit begs for understanding, to learn the behavior and boundaries of the perpetual change that is life. The GRN presented here are inferred using steady-state mechanics. Many factors are overlooked in the name of simplicity (see \cref{sec:regwhatwhen}, least of which is cell cycle synchronization, to measure separate experiments through similar periods of cell development and presumably under similar GRN regulation. This can be accomplished through starvation, \eg media deprivation forcing cells to recover from same stimuli together and thus bring their growth phases into sync. Such a uniforming starting point would allow time-series experimental quantification beyond simple assumptions of steady-state. This would enable modeling of relationship permanency, \ie if links are static, or more likely, if links are as transient as the condition the living systems finds itself inhabiting. This is one of the capabilities we have witnessed in the vastness of new high dimensional public datasets, \eg the LINCS project \citep{subramanian2017next}. Its high-throughput multiplexing has many time points taken within cell lines, which can feed into a model of network evolution, how links come into being, and thus a potential roadmap for which network conditions preclude any given network outcome.
%%drugL1000
The LINCS consortium also provides in its L1000 portal small molecule and drug induced perturbation experimental measures in keeping with the shRNA measures (see \cref{sec:ODE}) we use in \textbf{Paper IV}. It would be very interesting to develop a method for modeling drug perturbation on a network template, which could then be cross-correlated with inverse regulatory interactions in disease. Such pairing of disease and drug information in the context of networks could provide not only novel drug gene-targets, but also open the ability of less specific drugs targeting general network modules to achieve the same aim of changing the regulation of key disease pathways.

One straightforward method to achieve this using the similar models as those implemented here would be to code perturbation design onto small molecule, drug induced perturbation by way of the STITCH  \citep{kuhn2009stitch} or DrugBank \citep{wishart2006drugbank} databases. Creating such network models based purely on replicate, multiple perturbation experiments could uncover novel uses for drugs which have passed FDA phase I and II safety testing, thereby bypassing many years of safety testing not to mention chemical/structural development \citep{oprea2011drug}. These could then be very easily validated, at least in a rudimentary way through IC50 assay (50\% inhibitory concentration test) on the various standardized cell lines.
%%NN inf
Beyond perturbation-based inference methods, classic perception neural networks may hold key to further enhancing inference accuracy \citep{grimaldi2011regnann}, \citep{hache2007reconstruction}. Assembling a few layers to account for ingoing and outgoing links, strength as weights and up or down regulation would be easy when matched with a bank of sufficient training data. GeneSPIDER (\textbf{Paper I}) could be such a key, allowing for the creation of sufficient synthetic data for training with hold-out allowing for accuracy evaluation before feeding in real biological data. Furthermore, a bootstrap method drawing from experimental replicates would increase the training data amount as well as feed in a level of robustness where a network can describe many datasets due to varying levels of noise, combinations of experiments.

% \noindent\rule{12cm}{0.4pt}
\noindent\hrulefill

These past years I have been inspired to study many of the concepts present herein without directly identifying them as such. Specifically, I can recall that over the past year I spent compiling these chapters I been inspired by various forms of pop culture. These works reinforced what might be obvious, that genes (or anything else for that matter) do not and cannot exist in isolation. Moreover, despite the scientific method all but requiring it, any study done in isolation limits the aims and/or scope of the very study being done. Only as I saw this idea present itself in different domains again and again did I come to understand the repercussions. I was first so inspiration through what I have referenced formally from Carlo Rovelli's Order of Time\citep{rovelli}, which I followed reading Nick Pyenson's Spying on Whales\citep{pyenson},  and then Richard Powers' The Overstory\citep{powers}. These cultural works compliment very well what my studies suggest, a position to the much belabored quandry ``nature vs nurture'', which would seemingly posit that, like the idea that time is an artificial construct of the human mind or the age-old question ``what is the meaning of life'', there can be no separation of the two, nature and nurture, and these are simply bad questions. More specifically, the challenge we strive to collectively overcome is modeling complexity in our natural world, whether it be turbulent airflows or disease \emph{in vitro}. Thus changes to the system should be minimized to ensure the system maintains as many of its initial wild-type properties as possible, \eg culturing practices no matter how stringent the protocol, alters the system under study. Similarly, studying a tree isolated from its forest brethren does not characterize the wild-type individual just as investigating an orphaned and abandoned whale calve in captivity fails to capture its true nature. The encompassing system is composed of individuals which are themselves reflections of the constraints dictated by the environment. More to the point, there is no suitable nature without a nurture for it to grow and thrive within, the two are infact two sides of the same coin, as previously alluded to in the very beginning of \cref{sec:intro}.

We have understood that genes can maintain function over evolutionary time in similar species, \ie functional persistence, which allows transfer of function between newly characterized species within this similarity constraint; however, as equally well know, outside these similar species the gene, protein, \etc can and may very well develop entirely new functions. This demonstrates just how important the environment is to the function of any biomolecule, and thus how crucial it is we faithfully model the interactions within that environment. Such investigation, not excluding or excusing that presented here, often fall short of generalizing to the question initially posited, calling upon statistical tests to determine meaningful insight gained from the abstractions they become; nevertheless humanity's best approaches do over time consistently yield understanding, and as the tide of progress is never completely forward, we must endeavor to push on.

% The contributions I have made to the field are: 1) the creation of GeneSPIDER an environment for comparing inference methods; 2) NestBoot, a framework for enhancing many inference methods’ accuracy; 3) BalanceFitError (BFE), a method for measuring accuracy of inferred networks when gold standards are unavailable. GeneSPIDER has developed into the benchmarking environment for the consequent two projects. The NestBoot method initially found only a marginal ability to increase inference accuracies by comparing the network overlap across bootstraps. It was redesigned to implement a more strict thresholding manner by forming a null network distribution from the inference of networks based upon shuffled data, lending to the ability of the NestBoot framework to infer accurate networks by defining a way to enforce a naive but conservative FDR threshold. Similarly, BFE aims to overcome the limitation of scoring inference accuracy when no gold standard network is available by developing another null. Unlike our NestBoot null, here GRN links are shuffled after the initial inference to form a null expected-link distribution by which to compare any inferred network’s link composition, allowing us to score the ability of an inferred network to explain data relative to an expected error defined by this null distribution.

% Networks are all around us. In fact, you may count yourself part to any number without realizing. The modern incantation of social networks as omnipresent and unavoidable realities is just one example of how relationships continue to form the basis for society, and viewed the right angle, for all life. One has only to propose a new perspective and surely relationships between constituent members will present themselves. Such members and their relationships can be summarized as network structure. When the interaction is obvious, forming a network requires little more that the will to compile. However, when the forces at work are a bit more esoteric, say in the realm of living physiology, witnessing the components is challenging enough, let alone characterizing their interplay. Yet this is precisely what is at the heart of this thesis, methods and tools the scientific community has devised and is devising to tease out relationships between the material of life, here namely genes.

% In what I see as the culmination of my PhD studies, I have spearheaded a fourth study, which leverages experimental replicates within a high dimensionality public dataset. Utilizing a less pre-processed version of the data and normalizing in such a way that does not pool replicate experiments, we feed these replicates into our NestBoot FDR-enforcing inference framework. This creates cell type specific GRN, contrasting which allows for the search of conserved links and even modules. My greater ambition for this and any GRN project is to expand and infer using more input data types, and perhaps estimating regulatory dynamics using several time point rather than the steady state our model assumes. Identifying modules common among or specific to any cancer subtype GRN structure could expand the search space when identifying targetable biomarkers. A practical, testable application of biomarker identification in this manner would then be the integration of non-specific, small molecule perturbation data to enable the identification of novel drug-disease matches, \ie repositioning. While this is an ongoing investigation and not included here, it is my ultimate ambition to infer GRNs on whole genome level.

% I have been inspired by the robust relationships present in our natural world, in the way forest root systems propagate between individuals to form reactionary defenses against invasive species and how whale pods can communicate long term relationships among individual members...\\

% Future work hopes to involve the akaike and/or bayesian information criterion (AIC/BIC) to further bolster our results away from any bias we may introduce in our current estimations of inference performance, etc.


% ---------------------------------------------------------------------------
%: ----------------------- end of thesis sub-document ------------------------
% ---------------------------------------------------------------------------

