% this file is called up by thesis.tex
% content in this file will be fed into the main document

%: ----------------------- introduction file header -----------------------

\graphicspath{{1_introduction/figures/}} % specifies where the figures are stored

% ----------------------------------------------------------------------
%: ----------------------- content ----------------------- 
% ----------------------------------------------------------------------



%: ----------------------- HELP: latex document organisation
% the commands below help you to subdivide and organise your thesis
%    \chapter{}       = level 1, top level
%    \section{}       = level 2
%    \subsection{}    = level 3
%    \subsubsection{} = level 4
% note that everything after the percentage sign is hidden from output

\setcounter{page}{1}
\chapter{Introduction}
\label{sec:intro}
It has been suggested that all organisms are \emph{Informavores} \citep{MARGALEF1996141}, in that they survive by consuming negative entropy. If this is indeed the case and life is preferred for its ability to increase entropy more quickly than non-life \citep{england2013statistical}, then surely the framework in which this information is stored, organized and communicated conveys much of its meaning, \ie nothing exists in isolation. The network of relations between disparate bits of information is a description of the information itself and as such must be accounted for to describe the information to some extent. Information is important in many if not all scientific disciplines, where the individual 0 or 1 is inconsequential if its position on the disk is lost, just as the mariner's ship could be variably priced dependent on what waters he navigated to, just as any gene's expression is only relevant in the context of the environment in which it was measured. As these examples typify, information has both an independent and dependent aspect. Investigation is often focused on the dependent aspect for its ability to be isolated; however, without proper context, framing any conclusion based on isolated investigation will lack generalizability, an argument I hope to reinforce in the coming sections.

Informatavores undoubtedly arose making use of some energy gradient to gather favorable molecules for some advantage, ultimately to birth longer, more stable molecules. Relationships between molecules can be summarized as a \emph{network}, where nodes are molecules and the edges or links connecting the network are favorable and unfavorable interactions among the molecules. Such a network would contain many of basic physiochemical principles of nature, conveying widely relational guidelines in a humanly interpretable manner. Indeed, any such interplay can be organized into such node and link relationships \citep{barabasi2004network}.

In the context of living systems, where such relationships may be essential for survival, a network reflects the necessity of the response it encodes, essentially a reflection of its environment, \eg \coli quorum sensing towards increasing concentration of lactose when local glucose is depleted. Relationships may develop consequently rigid and robust to ensure contact between elements encoding crucial response \eg always tumble towards glucose \citep{berg2000motile}. Others could develop to be flexible and highly intermittent if less essential \eg only tumble towards lactose if glucose is low.
These "survival" parameters are coded into the network, making the composition not only a structural rulebook of how to respond to stimuli but also a guidebook for other options if an initial response fails. 
Life utilizes this robust flexibility for means of survival when conditions change and adaptation is necessary, and again when betting against a second change back to the original state could be as deadly as not adapting in the first place. Uncovering these abilities our natural world has developed over some four billion years is a monumental task; yet many techniques have been and are being developed to do just that!

The natural sciences are bound by their shared ambition to uncover the relationships and other such living principles dictated by the fundamental forces of natural. Modern research in chemistry and physics developed concurrently with the advent of modern computation, as the questions they posed birthed problems whose solutions necessitated such computation, which was only later adapted in biological research. Classically, biologists have attempted to piece together the puzzle of life's perpetuation process through isolated cause and effect investigation, drawing upon whichever perturbation most obviously related to an observable, often phenotypic effect. Not for this reason alone, biological study has lagged behind its peers in adapting to analytic methods of study in our modern computational world. The advent of reading protein primary structure follow by sequencing DNA spawned algorithms of comprehension, deriving information from the new found data. Thus began the cycle of breakthroughs leading to wisdom by gathering ever more knowledge, \ie accruing understanding. This recently birthed high throughput quantification methods and analytic techniques, a sort of biomolecular quantification which form the foundation of much of our modern understanding. This understanding dictates, among many other things, that much as using light to observe electron activity alters that very system, so too isolation of genes from a native system deviates any behavior from its natural tendency. Ceding to yet more currently insurmountable limitations such as the artificial growth environment and media, the systems approach to biological inquiry aims to more accurately characterize intracellular relationships by characterizing the cell-wide regulatory behavior at once based on characterization of the whole. Many tools exist within the subdomain to deduce relationships, all tuned to exploit certain aspects of any given experimental setup while largely overlooking limitations.

A major focus of the research presented in this thesis is the undertaking of perturbation biology, wherein one pokes and prods the (sub)cellular environment systematically to gain information to its capability for robust response. Specifically the network inference process attempts to define biological regulatory mechanisms separately from current limitations in the characterization process. Differentiating between these two is the ultimate goal, relegating the former to the signal and the later to systemic noise, thereby defining and hopefully limiting the role of noise in this reverse engineering process. Conditions under which any certain method would be more advantageously applied than another are highlighted and offer a guide for more actionable gene regulatory networks (GRN). 

A major portion of this work is an extension of work done in \cite{Nordling2013} and \cite{Tjärnberg2015etb}. Rather than rehash details presented therewithin each of these works, it is my intent to expound upon several key ideas formalized in each work based on new finding since their publication. This is demonstrated most plainly in what I see as the culmination of many if not all these ideas, theoretical and practical alike, in the form of \textbf{Paper III}.

% The contributions I have made to the field are: 1) the creation of GeneSPIDER an environment for comparing inference methods; 2) NestBoot, a framework for enhancing many inference methods’ accuracy; 3) BalanceFitError, a method for measuring accuracy of inferred networks when gold standards are unavailable. GeneSPIDER has developed into the benchmarking environment for the consequent two projects. The NestBoot method initially found only a marginal ability to increase inference accuracies by comparing the network overlap across bootstraps. Thus I, along with my coauthor team, designed and implemented a more strict thresholding manner by forming a null network distribution from the inference of networks based upon shuffled data. This lends to the ability of the NestBoot framework to infer accurate networks by defining a way to enforce a naive but conservative FDR threshold. Similarly, our most recent contribution aims to overcome the limitation of scoring inference accuracy when no gold standard network is available by developing another null distribution by which to compare. Unlike our NestBoot null, here GRN links are shuffled after the initial inference to form a null expected-link distribution by which to compare any inferred network’s link composition. This allows us to score the ability of an inferred network to explain data relative to an expected error defined by this null distribution. 

% In what I see as the culmination of my PhD studies, I have spearheaded a fourth study, which leverages experimental replicates within a high dimensionality public dataset. Utilizing a less pre-processed version of the data and normalizing in such a way that does not pool replicate experiments, we feed these replicates into our NestBoot FDR-enforcing inference framework. This creates cell type specific GRN, contrasting which allows for the search of conserved links and even modules. My greater ambition for this and any GRN project is to expand and infer using more input data types, and perhaps estimating regulatory dynamics using several time point rather than the steady state our model assumes. Identifying modules common among or specific to any cancer subtype GRN structure could expand the search space when identifying targetable biomarkers. A practical, testable application of biomarker identification in this manner would then be the integration of non-specific, small molecule perturbation data to enable the identification of novel drug-disease matches, \ie repositioning. %While this is an ongoing investigation and not included here, it is my ultimate ambition to infer GRNs on whole genome level.



%\subsection{Name your subsection} % subsection headings are again smaller than section names
%\subsubsection{Name your subsubsection} % subsubsection headings are again smaller than subsection names








%Starch of plants and glycogen of animals consists of $\alpha$-1,4-glycosidic glucose polymers \citep{lastname07}. See figure \ref{largepotato} for a comparison of glucose polymer structure and chemistry. 

%Two references can be placed separated by a comma \citep{lastname07,name06}.

%: ----------------------- HELP: references
% References can be links to figures, tables, sections, or references.
% For figures, tables, and text you define the target of the link with \label{XYZ}. Then you call cross-link with the command \ref{XYZ}, as above
% Citations are bound in a very similar way with \citep{XYZ}. You store your references in a BibTex file with a programme like BibDesk.





%\figuremacro{largepotato}{A common glucose polymers}{The figure shows starch granules in potato cells, taken from \href{http://molecularexpressions.com/micro/gallery/burgersnfries/burgersnfries4.html}{Molecular Expressions}.}

%: ----------------------- HELP: adding figures with macros
% This template provides a very convenient way to add figures with minimal code.
% \figuremacro{1}{2}{3}{4} calls up a series of commands formating your image.
% 1 = name of the file without extension; PNG, JPEG is ok; GIF doesn't work
% 2 = title of the figure AND the name of the label for cross-linking
% 3 = caption text for the figure

%: ----------------------- HELP: www links
% You can also see above how, www links are placed
% \href{http://www.something.net}{link text}

	%\figuremacroW{largepotato}{Title}{Caption}{0.8}

% variation of the above macro with a width setting
% \figuremacroW{1}{2}{3}{4}
% 1-3 as above
% 4 = size relative to text width which is 1; use this to reduce figures




%: ----------------------- HELP: tables
% Directly coding tables in latex is tiresome. See below.
% I would recommend using a converter macro that allows you to make the table in Excel and convert them into latex code which you can then paste into your doc.
% This is the link: http://www.softpedia.com/get/Office-tools/Other-Office-Tools/Excel2Latex.shtml
% It's a Excel template file containing a macro for the conversion.

%\begin{table}[htdp]
%\centering
%\begin{tabular}{ccc} % ccc means 3 columns, all centered; alternatives are l, r

%{\bf Gene} & {\bf GeneID} & {\bf Length} \\ 
% & denotes the end of a cell/column, \\ changes to next table row
%\hline % draws a line under the column headers

%human latexin & 1234 & 14.9 kbps \\
%mouse latexin & 2345 & 10.1 kbps \\
%rat latexin   & 3456 & 9.6 kbps \\
% Watch out. Every line must have 3 columns = 2x &. 
% Otherwise you will get an error.

%\end{tabular}
%\caption[title of table]{\textbf{title of table} - Overview of latexin genes.}
% You only need to write the title twice if you don't want it to appear in bold in the list of tables.
%\label{latexin_genes} % label for cross-links with \ref{latexin_genes}
%\end{table}



% There you go. You already know the most important things.


% ----------------------------------------------------------------------



