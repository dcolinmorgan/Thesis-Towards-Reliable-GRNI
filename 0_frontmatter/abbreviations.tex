% this file is called up by thesis.tex
% content in this file will be fed into the main document

% Glossary entries are defined with the command \nomenclature{1}{2}
% 1 = Entry name, e.g. abbreviation; 2 = Explanation
% You can place all explanations in this separate file or declare them in the middle of the text. Either way they will be collected in the glossary.

% required to print nomenclature name to page header
\markboth{\MakeUppercase{\nomname}}{\MakeUppercase{\nomname}}


% ----------------------- contents from here ------------------------
\nomenclature{AUPR}{Area Under Precision Recall}.
\nomenclature{AUROC}{Area Under Receiver Operating Characteristic}.
% \nomenclature{CDF}{cumulative density function}.
\nomenclature{CLS}{Constrained Least Squares}.
\nomenclature{CV}{Cross Validation in the form of leave one out (\textbf{LOO})}.
\nomenclature{DAG}{Directed Acyclic Graphs}.
\nomenclature{FBL}{Feedback Loop}.
% \nomenclature{FN}{false positive}.
% \nomenclature{FP}{false negative}.
% \nomenclature{GGM}{graphical Gaussian model}.
\nomenclature{GRN}{gene Regulatory Network, often coupled with inference (\textbf{GRNI})}.
\nomenclature{IAA}{InterAmpAtteness, otherwise known as Condition}.
\nomenclature{KD}{Knock-Down}.
\nomenclature{LASSO}{Least Absolute Shrinkage and Selection Operator}.
\nomenclature{LSCO}{Least Squares with CutOff, also of the total LSCO variant (\textbf{TLSCO})}.
% \nomenclature{LSCO}{least squares with CutOff}.
\nomenclature{MI}{Mutual Information}.
\nomenclature{MCC}{Matthew's Correlation Coefficient}.
\nomenclature{ODE}{Ordinary Differential Equation, here of the first order \& linear variety}.
\nomenclature{RNI}{Robust Network Inference (with cutoff (\textbf{RNICO})}.
\nomenclature{wRSS}{weighted Residual Sum of Squares}.
\nomenclature{SNR}{Rignal-to-Roise Ratio}.
% \nomenclature{TLSCO}{total least squares with cutoff}.
% \nomenclature{TN}{true negative}.
% \nomenclature{TP}{true positive}.

\nomenclature{$\alpha$}{Sparsity parameter}.
\nomenclature{$\theta{X}$}{true X}.
\nomenclature{$\norm{X}$}{norm of X, often of the Frobenius variant $\norm{X}^2$}.
\nomenclature{$\hat{X}$}{estimator of X}.
\nomenclature{$\zeta$}{Regularization parameter}.

%
