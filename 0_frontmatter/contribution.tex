% Author's contribution -----------------------------------------------------

% \chapter{Author's contribution}

% I joined the group very much in the middle of large projects, all revolving around perturbation based wet-lab experiment, somewhat of a rarity for our theory focused group. I helped finalize the qPCR for the experiment, and helped debug a bit of the main GeneSPIDER code. I then ran the benchmark for GeneSPIDER as well as lead the Nested Bootstrap, Perturbation-based Inference and worked with Deniz to finalize the large scale perturbation-based inference and provided minor input to her finishing the benchmark.
\vspace*{\fill}



''To begin with, the art of jigsaw puzzles seems of little substance, easily exhausted, wholly dealt with by a basic introduction to Gestalt: the perceived object -- we may be dealing with a perceptual act, the acquisition of a skill, a physiological system, or, as in the present case, a wooden jigsaw puzzle -- is not a sum of elements to be distinguished from each other and analysed discretely, but a pattern, that is to say a form, a structure: the element's existence does not precede the existence of the whole, it comes neither before nor after it, for the parts do not determine the pattern, but the pattern determines the parts: knowledge of the pattern and of its laws, of the set and its structure, could not possibly be derived from discrete knowledge of the elements that compose it. That means that you can look at a piece of a puzzle for three whole days, you can believe that you know all there is to know about its colouring and shape, and be no further on than when you started. The only thing that counts is the ability to link this piece to other pieces, and in that sense the art of the jigsaw puzzle has something in common with the art of go. The pieces are readable, take on a sense, only when assembled; in isolation, a puzzle piece means nothing -- just an impossible question, an opaque challenge. But as soon as you have succeeded, after minutes of trial and error, or after a prodigious half-second flash of inspiration, in fitting it into one of its neighbours, the piece disappears, ceases to exist as a piece. The intense difficulty preceding this link-up -- which the English word puzzle indicates so well -- not only loses its raison d'$\hat{e}$tre, it seems never to have had any reason, so obvious does the solution appear. The two pieces so miraculously conjoined are henceforth one, which in its turn will be a source of error, hesitation, dismay, and expectation.''
\\
\begin{flushright}
--Georges Perec, La Vie mode d'emploi (1978)
\end{flushright}
\vspace*{\fill}
% ---------------------------------------------------------------------- 
