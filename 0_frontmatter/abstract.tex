% Thesis Abstract -----------------------------------------------------

\begin{abstracts}

Phenotypic traits are now known to stem from the interplay between genetic variables across many if not every level of biology. The field of gene regulatory network (GRN) inference is concerned with understanding the regulatory interactions between genes in a cell, in order to build a model that captures the behaviour of the system. Perturbation biology, whereby genes or RNAs are targeted and their activity altered, has enabled the GRN field to model entire webs of \textbf{influence}. By first systematically perturbing the system and then reading the system's reaction as a whole, we can feed this data into various methods to reverse engineer the key agents of change. For example, cancer is known to stem from multiple, independent mutations, the effects of which aggregate to gain control of cellular activity. Much study focuses on isolated mutations seen to be crucial markers of disease progression. However, this forfeits a greater sense of any single gene's causative role in the developing systemic flux. By way of examining the interrelatedness of the system's components, such inference offers a more meaningful understanding and thereby the tools to target progression from the wild-type state.

% The initial study sets the groundwork for the rest, and deals with finding common ground among the sundry methods in order to compare and rank performance in an unbiased setting. The GeneSPIDER MATLAB package is an inference benchmarking platform whereby methods can be added via a wrapper for testing in competition with one another. Synthetic datasets and networks spanning a wide range of conditions can be created for this purpose. The evaluation of method across various conditions in the benchmark therein demonstrates which properties influence the accuracy of which methods, and thus which are more suitable for use under any given characterized condition.

% The second study introduces a novel framework for increasing inference accuracies within the GS environment by independent, nested bootstraps, \ie repeated inference trials. Under low to medium noise levels, this allows support to be gathered for links occurring most often while spurious links are discarded through comparison to an estimated null, shuffled-link distribution. While noise continues to plague every method, nested bootstrapping in this way is shown to increase the accuracy of several different methods.

% The third study is a small-scale test of these components on real data, which finds a reliable network for a dataset covering 40 genes perturbed in a human squamous carcinoma cell line. The methods of inference are again wrapped with NestBoot so to contain links of high support, and indeed these networks are more accurate on average than those of networks of shuffled topologies. A network of high confidence was recovered containing many links known to the literature, as well as a slew of novel links.A final study takes aim at inferring GRN on the genome-wide scale.

% The final study breaks from the restrictions of the synthetic datasets of the first two and the small scale of the third study to infer reliable networks on large scale, public perturbation data. Utilizing many biological datasets of a far greater scale here we seek to firstly isolate signal and secondly differentiate the networks based tissue subtype. We hope to demonstrate some semblance of a core network module necessary for disease development through quite a basic relatedness comparison but over large enough a scale to bring about insight into novel mechanisms of cancer progression.

The contributions I have made to the field are: 1) the creation of GeneSPIDER an environment for comparing inference methods; 2) NestBoot, a framework for enhancing many inference methods’ accuracy; 3) BalanceFitError, a method for measuring accuracy of inferred networks when gold standards are unavailable. GeneSPIDER has developed into the benchmarking environment for the consequent two projects. The NestBoot method initially found only a marginal ability to increase inference accuracies by comparing the network overlap across bootstraps. Thus I, along with my coauthor team, designed and implemented a more strict thresholding manner by forming a null network distribution from the inference of networks based upon shuffled data, lending to the ability of the NestBoot framework to infer accurate networks by defining a way to enforce a naive but conservative FDR threshold. SImilarly, our most recent submission aims to overcome the limitation of scoring inference accuracy when no gold standard network is available by developing another null. Unlike our NestBoot null, here GRN links are shuffled after the initial inference to form a null expected-link distribution by which to compare any inferred network’s link composition. This allows us to score the ability of an inferred network to explain data relative to an expected error defined by this null distribution. %In what I see as the culmination of my PhD studies, I have spearheaded a fourth study, which leverages experimental replicates within a high dimensionality public dataset. 


\end{abstracts}

% ---------------------------------------------------------------------- 
